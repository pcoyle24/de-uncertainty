\documentclass[11pt]{article}
\usepackage{scrextend}
\usepackage{amssymb}
\usepackage{amsfonts}
\usepackage{amsmath}
\usepackage{mathtools}
\usepackage{bm}

%\usepackage{dsfont}
% \usepackage{bbm}

\usepackage[nohead]{geometry}
\usepackage[onehalfspacing]{setspace}
\usepackage[bottom]{footmisc}
\usepackage{indentfirst}
\usepackage{endnotes}
\usepackage{mathtools}

\usepackage{graphicx}
\usepackage{graphics}
\usepackage{epstopdf}

%\usepackage{epsfig}

\usepackage{lscape}
\usepackage{titlesec}
\usepackage{array}

%\usepackage{hyperref}

\usepackage{flexisym}
\usepackage{nccfoots}
\usepackage{datetime}
\usepackage{multirow}
\usepackage{booktabs}
\usepackage{rotating}

\usepackage{amsthm}
\newtheorem{theorem}{Theorem}
\newtheorem{definition}{Definition}
\newtheorem{remark}{Remark}
\newtheorem{proposition}{Proposition}
\newtheorem{corollary}{Corollary}[proposition]
\newtheorem{lemma}{Lemma}[theorem]

\usepackage[usenames,dvipsnames]{color}

\usepackage[longnamesfirst]{natbib}
\usepackage[justification=centering]{caption}

%\usepackage{datetime}

\DeclarePairedDelimiter\abs{\lvert}{\rvert}%
\DeclarePairedDelimiter\norm{\lVert}{\rVert}%

\definecolor{darkgray}{gray}{0.30}

%\usepackage[dvips, colorlinks=true, linkcolor=darkgray,

\usepackage[colorlinks=true, linkcolor=darkgray, citecolor=darkgray, urlcolor=darkgray, bookmarks=false, ,
pdfstartview={FitV},
pdftitle={Zero Bound Risk},
pdfauthor={Taisuke Nakata},
pdfkeywords={Liquidity Trap, Zero Lower Bound}]{hyperref}
\usepackage{subfig}
\usepackage{xcolor,colortbl}
\usepackage{float}

\newcommand*{\LargerCdot}{\raisebox{-.5ex}{\scalebox{2}{$\cdot$}}}

\makeatletter
\def\@biblabel#1{\hspace*{-\labelsep}}
\makeatother
\geometry{left=1.2in,right=1.2in,top=1in,bottom=1in}


\begin{document}
	
	\title{Deflationary Equilibrium under Uncertainty\footnote{We thank x and y for useful suggestions. The views expressed in this paper, and all errors and omissions, should be regarded as those of the authors, and are not necessarily those of the Federal Reserve Board of Governors or the Federal Reserve System.}}
	\author{
		Philip Coyle\thanks{Board of Governors of the Federal Reserve System, Division of Research and Statistics, 20th Street and Constitution Avenue N.W. Washington, D.C. 20551; Email: philip.m.coyle@frb.gov.}\\
		Federal Reserve Board
		\and 
		Taisuke Nakata\thanks{Board of Governors of the Federal Reserve System, Division of Research and Statistics, 20th Street and Constitution Avenue N.W. Washington, D.C. 20551; Email: taisuke.nakata@frb.gov.}\\
		Federal Reserve Board
		\and Sebastian Schmidt\thanks{%
			European Central Bank, Monetary Policy Research Division, 60640 Frankfurt, Germany; Email: sebastian.schmidt@ecb.int}\\
		European Central Bank
	}
	%\newdateformat{mydate}{First Draft: March 2019\\This Draft: \monthname[\THEMONTH] \THEYEAR}
	\newdateformat{mydate}{This Draft: \monthname[\THEMONTH] \THEYEAR}
	\date{\mydate\today}
	
	\maketitle
	
	\vspace{-0.3in}
	
	\begin{center}
		\textbf{Abstract}
	\end{center}
	\noindent We analytically and numerically demonstrate that the so-called deflationary equilibrium of the New Keynesian model may feature a positive---albeit below the target rate---inflation at the (risky) steady state. Necessary and sufficient conditions for inflation to be positive in the deflationary steady state are (i) the degree of uncertainty is high and (ii) the inflation target is positive. We also demonstrate that the steady state interest rate in the deflationary equilibrium is positive, regardless of the inflation target, if uncertainty is sufficiently large. Our analysis suggests that a persistently positive inflation rate or interest rate does not necessarily mean that the economy has escaped the deflationary equilibrium. 
	
	\vspace{5em}
	
	\noindent JEL: E32, E52, E61, E62, E63\\
	
	\noindent Keywords: Liquidity Traps, Deflationary Equilibrium, Zero Lower Bound.
	
	\newpage
	
	\section{Introduction}
	\label{S:Introduction}
	
	
	%==========================================================
	%==========================================================
	%==========================================================
	%==========================================================
	%==========================================================
	%==========================================================
	%==========================================================
	%==========================================================
	%========================================================== Model (Stylized Model)
	%==========================================================
	%==========================================================
	%==========================================================
	%==========================================================
	%==========================================================
	%==========================================================
	%==========================================================
	%\section{Stylized Model}
	\section{Model}
	\label{S:Model}
	
	This section presents the baseline model, defines the policy rule employed by the central bank, and defines the equilibrium. We employ a standard New Keynesian model formulated in discrete time with an infinite horizon, as is consistent with \citet{Woodford2003Book} and \citet{Gali2015Book}. To derive some analytical results, we put all model equations in a semi log-linear form.
	
	The equilibrium conditions of model are given by the following system of equations: 
	\begin{align}
		& y_{t} = \mathbb{E}_t\{y_{t+1}\} - \sigma\left[\hat{i}_t - \mathbb{E}_t\{\hat{\pi}_{t+1}\}-r_t^n\right] \label{eq:EE}\\
		& \hat{\pi}_{t} = \kappa y_t + \beta\mathbb{E}_t\{\hat{\pi}_{t+1}\}\label{eq:PC}\\
		& \hat{\pi}_t = \pi_t - \pi^*\\
		& \hat{i}_t = i_t - i_{DSS} \\
		& i_t = \text{max}\left[0,r^* + \pi^* + \phi_{\pi}(\pi_t - \pi^*)\right]\label{eq:TR}\\
		& i_{DSS} = r^* + \pi^*
	\end{align}
	$y_t, \pi_t, i_t$ are the output gap, inflation rate, and nominal interest rate of our economy. Equation (\ref{eq:EE}) is the consumption Euler equation, equation (\ref{eq:PC}) is the standard New Keynesian Phillips Curve, and equation (\ref{eq:TR}) is the interest-rate feedback rule followed by the central bank. 
	
	The parameters are defined as follows: $\beta\in(0, 1)$ denotes the representative household's subjective discount factor; $\sigma>0$ is the intertemporal elasticity of substitution in consumption; $\kappa$ represents the slope of the New Keynesian Phillips Curve; $r^*$ is the natural rate of interest; $\pi^*$ is the inflation target; $\phi_{\pi}$ is the multiplier on inflation in the Taylor Rule. 
	
	For our baseline model, we assume our demand shock, $r_t^n$ follows a three-state i.i.d. shock. In particular, $r_t^n$ takes the value of either $r_H^n = r^n$ in the high (non-crisis) state, $r_M^n = 0$ in the middle state, or $r_L^n = -r^n$ in the low (crisis) state. Because our shock follows an i.i.d. process, the transition probabilities are given by the following: 
	
	\begin{align}
		& \text{Prob}\left(r_{t+1}^n = r_L^n | r_t^n \right) = \frac{1-p_M}{2}\\
		& \text{Prob}\left(r_{t+1}^n = r_M^n | r_t^n \right) = p_M\\
		& \text{Prob}\left(r_{t+1}^n = r_H^n | r_t^n \right) = \frac{1-p_M}{2}
	\end{align}
	\noindent
	where $p_M$ is the probability of moving to the middle state in the next period. 
	
	The standard equilibrium is given by a vector $\{y_H,\pi_H,i_H,y_M,\pi_M,i_M,y_L,\pi_L,i_L\}$ that solves the following system of equations: 
	\begin{align}
		y_{H} & = \mathbb{E}\{y_{t+1}\} - \sigma\left[i_{H} - i_{DSS} - \mathbb{E}\{\pi_{t+1} - \pi^*\}-r^n\right] \label{eq:EE_H} \\
		y_{M} & = \mathbb{E}\{y_{t+1}\} - \sigma\left[i_{M} - i_{DSS} - \mathbb{E}\{\pi_{t+1} - \pi^*\}\right] \label{eq:EE_M}\\
		y_{L} & = \mathbb{E}\{y_{t+1}\} - \sigma\left[i_{L} - i_{DSS} - \mathbb{E}\{\pi_{t+1} - \pi^*\}+r^n\right] \label{eq:EE_L}\\
		\pi_{H} - \pi^* & = \kappa y_{H} + \beta\mathbb{E}\{\pi_{t+1} - \pi^*\} \label{eq:PC_H}\\
		\pi_{M} - \pi^* & = \kappa y_{M} + \beta\mathbb{E}\{\pi_{t+1} - \pi^*\} \label{eq:PC_M}\\
		\pi_{L} - \pi^* & = \kappa y_{L} + \beta\mathbb{E}\{\pi_{t+1} - \pi^*\} \label{eq:PC_L}\\
		i_{H} & = \text{max}\left[0, r^* + \pi^* +  \phi_{\pi}(\pi_{H} - \pi^*)\right] \label{eq:TR_H}\\
		i_{M} & = \text{max}\left[0, r^* + \pi^* + \phi_{\pi}(\pi_{M} - \pi^*)\right] \label{eq:TR_M}\\
		i_{L} & = \text{max}\left[0, r^* + \pi^* + \phi_{\pi}(\pi_{L} - \pi^*)\right] \label{eq:TR_L}
	\end{align}
	where $\mathbb{E}\{x_{t+1}\} \coloneqq \frac{1-p_M}{2}x_H + p_Mx_M + \frac{1-p_M}{2}x_L$ and $x\in\{y,\pi\}$.
	%Note that the more transitory the expectations-driven LT is, the lower the output gap during the expectations-driven LT, as discussed in Appendix~\ref{A:pTpD}.
	\section{Risk-Adjusted Fisher Relation}
	
	One way to understand the discrepancy between the deterministic and risky steady states is to examine the effect of uncertainty on the Fisher relation. In an environment with no uncertainty, there exists a relationship between middle state inflation ($\pi_M^{DSS}$) and middle state nominal interest rate ($i_M^{DSS}$). This relationship is obtained by evaluating equation (\ref{eq:EE_M}) at its deterministic steady state and becomes
	\begin{align}
		& i_M^{DSS} =  r^* + \pi_M^{DSS} \label{eq:FR}
	\end{align}
	after dropping the expectational operator, and eliminating the deterministic steady-state consumption from both sides of the equation. Equation (\ref{eq:FR}), graphically seen in the blue line of figure (\ref{fig:RAFR}) is often referred to as the Fisher relation. The intersection of the standard Fisher relation and the Taylor rule (solid black line) are the deterministic steady states of the model. The presence of the ELB on interest rates creates two points where the standard Fisher relation and Taylor rule intersect, as first pointed out in \citet{BenhabibSchmittGroheUribe2002}. The first deterministic steady state is one where inflation rates are non-negative and at target, and the policy rate is above zero: this is known as the target state. The second deterministic steady state, known as the deflationary state, occurs when interest rates are at the ELB and inflation is $-r^*$. 
	
	\begin{figure}[!ht]
		\begin{center}
			\caption{Risk Adjusted Fisher Relation}
			\includegraphics[width = 12cm ]{Figs/Fig1/RAFR.eps}\label{fig:RAFR}
		\end{center}
	\end{figure}
	
	In the stochastic environment, the risk-adjusted Fisher relation diverges at points from the standard Fisher relation due to high amounts of uncertainty. The risk-adjusted Fisher relation is similarly obtained from evaluating equation (\ref{eq:EE_M}) at the risky steady state and rearranging for middle state interest rate: 
	\begin{align*}
		i_M^{RSS} &= \sigma^{-1}\left(\mathbb{E}_{RSS}\{y_{t+1}\} - y_M^{RSS}\right) + r^* + \mathbb{E}_{RSS}\{\pi_{t+1}\} 
	\end{align*}
	If we add and subtract $\pi_M^{RSS}$ on the right hand side, we have our risk-adjusted Fisher relation:
	\begin{align}
		i_M^{RSS} &= r^* + \pi_M^{RSS} + \sigma^{-1}\left(\mathbb{E}_{RSS}\{y_{t+1}\} - y_M^{RSS}\right) + \left(\mathbb{E}_{RSS}\{\pi_{t+1}\} - \pi_M^{RSS}\right) \label{eq:RAFR}
	\end{align}
	Relative to the standard Fisher relation, there is an adjustment term ($\sigma^{-1}\left(\mathbb{E}_{RSS}\{y_{t+1}\} - y_M^{RSS}\right) + \left(\mathbb{E}_{RSS}\{\pi_{t+1}\} - \pi_M^{RSS}\right) $) that reflects the discrepancy between todays economic conditions and the expected economic conditions in the next period. This adjustment term captures the effects of uncertainty of future economic conditions differing from their current conditions induced by the possibility that the ELB may or may not bind in the future \footnote{Uncertainty is defined such that there is a difference in the ELB binding in all three states. For example, if the ELB does or does not bind in the high, medium, and low state, there is no uncertainty present in the economy. This is when the standard and risk-adjusted Fisher relation are equal. If the ELB were to only bind in the low state, while not binding in the medium or high state, however, then there is uncertainty present in the model and the risk-adjusted Fisher relation is below the standard Fisher relation.}. This adjustment term reflects the divergence of the risk-adjusted Fisher relation from the standard Fisher relation, which is graphically captured by the dotted red line in figure (\ref{fig:RAFR}). In our three-state model, table (\ref{t:Uncertainty}) illustrates that two cases of uncertainty exist. 
	
	\begin{table}
		\centering
		\begin{tabular}{ccc}
			\hline
			\hline
			States where & States where  & Presence of \\
			ELB Binds & ELB Does Not Bind & Uncertainty\\
			\hline
			L, M, H & None & No\\
			L, M & H & Yes \\
			L & M, H & Yes \\
			None & L, M, H & No\\
			\hline
			\hline
		\end{tabular}
		\caption{Calibrations}
		\label{t:Uncertainty}
	\end{table}
	
	While not immediately obvious,the risk-adjusted Fisher relation can be expressed as a function of $\pi_M$. This fact allows us to build out figure (\ref{fig:RAFR}): for every input $\pi_M$ we find the appropriate value of $i_M$ and leads us to our first proposition:
	
	\begin{proposition}
		The risk-adjusted Fisher relation can be written as a piecewise function of $\pi_M$.
	\end{proposition}
	\begin{proof}
		The algebraic derivations can be found in Appendix \ref{A:ThreeStateiid}. Observe that the risk-adjusted Fisher relation can be written as a piecewise function: 
		\begin{align}
		i_M = 
			\begin{cases}
			r^* + \pi_M & \text{if } \pi_M < \pi_{LB}\\
			r^* + \pi_M + \sigma^{-1}\left(\mathbb{E}_{LM}\{y_{t+1}\} - y_M^{LM}\right) + \left(\mathbb{E}_{LM}\{\pi_{t+1}\} - \pi_M\right) & \text{if } \pi_{LB} \le \pi_M \le \pi_{B}\\
			r^* + \pi_M + \sigma^{-1}\left(\mathbb{E}_{L}\{y_{t+1}\} - y_M^{L}\right) + \left(\mathbb{E}_{L}\{\pi_{t+1}\} - \pi_M\right) & \text{if } \pi_{B} \le \pi_M \le \pi_{UB}\\
			 r^* + \pi_M & \text{if } \pi_{UB} < \pi_M			
			\end{cases}
		\end{align}
		Where $x^{s}$, $s\in\{LM, L\}$ denotes the policy functions conditional on which states the ELB binds in. All policy functions and expectations are functions of $\pi_M$ and thus $i_M$ is a function of $\pi_M$. 
	\end{proof}
	
	%==========================================================
	%==========================================================
	%==========================================================
	%==========================================================
	%==========================================================
	%==========================================================
	%==========================================================
	%==========================================================
	%========================================================== Results (Stylized Model)
	%==========================================================
	%==========================================================
	%==========================================================
	%==========================================================
	%==========================================================
	%==========================================================
	%==========================================================
	\section{Analytical results from a model with a three-state shock}
	\label{S:AnalyticalResults}

	\subsection{Preliminary results}	
	We begin with a few definitions: 
	\begin{definition}
		$\pi_{B}$ is the value of inflation at which the Taylor Rule is identically equal to zero. In other words, it is the value that solves the following equation: $r^* + \pi^* + \phi_{\pi}(\pi_B - \pi^*) = 0$
	\end{definition}
	\begin{definition}
		The \underline{lower-bound cutoff ($\pi_{LB}$)} is the first point of the risk-adjusted Fisher relation that diverges from the standard Fisher relation on the domain $(-\infty, \pi_{B}]$.
	\end{definition}
	\begin{definition}
		The \underline{upper-bound cutoff ($\pi_{UB}$)} is the first point of the risk-adjusted Fisher relation that diverges from the standard Fisher relation on the domain $[\pi_{B},\infty)$
	\end{definition}
		
	\begin{proposition}\label{prop:cSHOCKS}
		There exists shock sizes $r^n_{LB}$, $r^n_{UB}$, and $r^n_{B}$ such that 
		\begin{enumerate}
			\item [i.] If $r^n = r^n_{LB}$ then $\pi_{LB} = \pi^{RSS}_{DR} = \pi^{DSS}_{DR}$ 
			\item [ii.] If $r^n = r^n_{UB}$ then $\pi_{UB} = \pi^{RSS}_{TR} = \pi^{DSS}_{TR}$
			\item [iii.] If $r^n = r^n_{B}$ then $\pi^{RSS}_{TR} = \pi_B$ or $\pi^{RSS}_{DR} = \pi_B$
		\end{enumerate}
	\end{proposition}
	\begin{proof}
		Equation (\ref{eq:RAFR}) can be rewritten in terms of $\pi_{LB}$ and $\pi_{UB}$:
		\begin{align*}
			\pi_{LB} &= \frac{\kappa\sigma r^n}{\kappa\sigma r^n + 1} - \pi_{B}\\
			\pi_{UB} &= \pi_{B} - \kappa\sigma r^n 	
		\end{align*}
		If we set $\pi_{LB} = \pi^{DSS}_{DR}$ and $\pi_{UB} = \pi^{DSS}_{TR}$ and solve for $r^n$, we have 
		\begin{align*}
			r^n_{LB} = r^n = &\left(\pi^*+r^* \right)\left(\phi_{\pi} - 1\right)\left(\kappa\phi_{\pi}\sigma\right)^{-1} \\
			r^n_{UB} = r^n = &\left(\pi^* + r^*\right) \left(\kappa\phi_{\pi}\sigma  + 1\right)\left(\kappa\phi_{\pi}\sigma\right)^{-1}\\
		\end{align*}
		To find the point where the RSS is equal to $\pi_{B} $, we set the two piecewise equations of the risk-adjusted Fisher relation equal to each other:
		\begin{align*}
			r^* + \pi_M + \sigma^{-1}\left(\mathbb{E}_{LM}\{y_{t+1}\} - y_M^{LM}\right) + \left(\mathbb{E}_{LM}\{\pi_{t+1}\} - \pi_M\right) \\ 
			= r^* + \pi_M + \sigma^{-1}\left(\mathbb{E}_{L}\{y_{t+1}\} - y_M^{L}\right) + \left(\mathbb{E}_{L}\{\pi_{t+1}\} - \pi_M\right)
		\end{align*}
		The intersection of these two lines is at $\pi_{B}$. If we isolate for $r^n$, when we get: 
		\begin{align*}
			r^n_{B} = r^n = &-\frac{2(\pi^* + r^*)(\phi_{\pi}-1)(\kappa\sigma\phi_{\pi}+1)} {\kappa\phi_{\pi}^2\sigma(\kappa\sigma+1)(p_M-1)}
		\end{align*}
	\end{proof}
	\begin{proposition}
		There does not exist a shock size $r^n_{LG} > \text{max}\left\{r^n_{LB}, r^n_{UB}, r^n_{B}\right\}$
	\end{proposition}
	\begin{proof}
		Assume, by way of contradiction, that such a shock exists. This would imply that the risk-adjusted Fisher relation does not come into contact with the Taylor rule, which violates our definition of the Risky Steady State. 
	\end{proof}	
	
	The above propositions provide a foundation for some of the analytical results to follow. Below, we provide some further results on $\text{max}\left\{r^n_{LB}, r^n_{UB}, r^n_{B}\right\}$ and the relationship to $\phi_{\pi}$. We again begin with a definition:
	\begin{definition}
		Let
		\begin{align*}
			\phi_{\pi,LB} = \frac{2}{p_M - \kappa\sigma(1-p_M) + 1}\\
			\phi_{\pi,UB} = \frac{-2}{p_M + \kappa\sigma(1+p_M) - 1}
		\end{align*}
	\end{definition}
	\begin{proposition}\label{prop:cPHIpiub}
		Let $\phi_{\pi}\in(1,\phi_{\pi,LB})$. Then $r^n_{UB} = \text{max}\left\{r^n_{LB}, r^n_{UB}, r^n_{B}\right\}$. 
	\end{proposition}
	\begin{proof}
		To begin, observe that if we evaluate $r^n_{LB}$, $r^n_{UB}$, and $r^n_{B}$ at $\phi_{\pi,LB}$ we get:
		\begin{align*}
			r^n_{LB} = -\frac{(\pi^* + r^*)(\kappa\sigma + 1)(p_M -1)}{2\kappa\sigma}\\
			r^n_{UB} =\frac{(\pi^* + r^*)(\kappa\sigma + 1)(p_M +1)}{2\kappa\sigma}\\
			r^n_{B} = \frac{(\pi^* + r^*)(\kappa\sigma + 1)(p_M +1)}{2\kappa\sigma}
		\end{align*}
		First, we will show that $r^n_{UB} > r^n_{LB}$. 
		Assume, by way of contradiction, not.  Then we have the following inequality: 
		\begin{align*}
			\frac{(\pi^* + r^*)(\kappa\sigma + 1)(p_M -1)}{2\kappa\sigma} &< -\frac{(\pi^* + r^*)(\kappa\sigma + 1)(p_M +1)}{2\kappa\sigma}\\
			(\pi^* + r^*)(\kappa\sigma + 1)(p_M -1) &< -(\pi^* + r^*)(\kappa\sigma + 1)(p_M +1)\\
			(p_M -1) &< -(p_M +1)
		\end{align*}
		This final statement implies that $0>2p_M$, implying that $p_M<0$. This yields a contradiction since $p_m>0$. To show that  $r^n_{UB} > r^n_{LB}$ for $\phi_{\pi}\in(1,\phi_{\pi,LB})$, we examine the derivatives $\phi_{\pi,LB}$: 
		\begin{align*}
			\frac{\partial r^n_{UB}}{\partial \phi_{\pi}} &= -\frac{(\pi^* + r^*)(p_M-\kappa\sigma(1+p_M)+1)^2}{4\kappa\sigma} < 0\\
			\frac{\partial r^n_{LB}}{\partial \phi_{\pi}} &= \frac{(\pi^* + r^*)(p_M-\kappa\sigma(1+p_M)+1)^2}{4\kappa\sigma} > 0\\
		\end{align*}
		Thus, for a decrease in  $\phi_{\pi}$, $r^n_{UB}$ continues to get larger than $r^n_{LB}$. 
		Next, observe that $r^n_{UB} = r^n_{B}$ at $\phi_{\pi,LB})$. To see that $r^n_{UB} > r^n_{B}$, we again look at the derivatives of $r^n_{UB}$ and $r^n_{B}$ evaluated at $\phi_{\pi,LB})$:
		\begin{align*}
			\frac{\partial r^n_{UB}}{\partial \phi_{\pi}} &= -\frac{(\pi^* + r^*)(p_M-\kappa\sigma(1+p_M)+1)^2}{4\kappa\sigma} < 0\\
			\frac{\partial r^n_{B}}{\partial \phi_{\pi}} &= -\frac{(\pi^* + r^*)p_M(p_M-\kappa\sigma(1+p_M)+1)^2}{2\kappa\sigma(p_M -1)} > 0\\
		\end{align*}
		Thus, for a decrease in  $\phi_{\pi}$, $r^n_{UB}$ gets larger than $r^n_{B}$. 
	\end{proof}
	\begin{proposition}\label{prop:cPHIpilb}
		Let $\phi_{\pi}\in(\phi_{\pi,UB},\infty)$. Then $r^n_{LB} = \text{max}\left\{r^n_{LB}, r^n_{UB}, r^n_{B}\right\}$. 
	\end{proposition}
	\begin{proof}
		To begin, observe that if we evaluate $r^n_{LB}$, $r^n_{UB}$, and $r^n_{B}$ at $\phi_{\pi,LB}$ we get:
		\begin{align*}
			r^n_{LB} =\frac{(\pi^* + r^*)(\kappa\sigma + 1)(p_M +1)}{2\kappa\sigma}\\
			r^n_{UB} =-\frac{(\pi^* + r^*)(\kappa\sigma + 1)(p_M -1)}{2\kappa\sigma}\\
			r^n_{B} = \frac{(\pi^* + r^*)(\kappa\sigma + 1)(p_M +1)}{2\kappa\sigma}
		\end{align*}
		First, we will show that $r^n_{LB} > r^n_{UB}$. 
		Assume, by way of contradiction, not.  Then we have the following inequality: 
		\begin{align*}
			-\frac{(\pi^* + r^*)(\kappa\sigma + 1)(p_M +1)}{2\kappa\sigma} &< \frac{(\pi^* + r^*)(\kappa\sigma + 1)(p_M -1)}{2\kappa\sigma}\\
			-(\pi^* + r^*)(\kappa\sigma + 1)(p_M +1) &< (\pi^* + r^*)(\kappa\sigma + 1)(p_M -1)\\
			(p_M -1) &< -(p_M +1)
		\end{align*}
		This final statement implies that $0>2p_M$, implying that $p_M<0$. This yields a contradiction since $p_m>0$ To show that  $r^n_{LB} > r^n_{UB}$ for $\phi_{\pi}\in(\phi_{\pi,UB},\infty)$, we examine the derivatives $\phi_{\pi,UB}$: 
		\begin{align*}
			\frac{\partial r^n_{UB}}{\partial \phi_{\pi}} &= -\frac{(\pi^* + r^*)(p_M-\kappa\sigma(1+p_M)+1)^2}{4\kappa\sigma} < 0\\
			\frac{\partial r^n_{LB}}{\partial \phi_{\pi}} &= \frac{(\pi^* + r^*)(p_M-\kappa\sigma(1+p_M)+1)^2}{4\kappa\sigma} > 0\\
		\end{align*}
		Thus, for an increase in  $\phi_{\pi}$, $r^n_{LB}$ continues to get larger than $r^n_{UB}$. 
		Next, observe that $r^n_{LB} = r^n_{B}$ at $\phi_{\pi,UB})$. To see that $r^n_{LB} > r^n_{B}$, we again look at the derivatives of $r^n_{LB}$ and $r^n_{B}$ evaluated at $\phi_{\pi,LB})$:
		\begin{align*}
			\frac{\partial r^n_{LB}}{\partial \phi_{\pi}} &= \frac{(\pi^* + r^*)(p_M-\kappa\sigma(1+p_M)+1)^2}{4\kappa\sigma} > 0\\
			\frac{\partial r^n_{B}}{\partial \phi_{\pi}} &= \frac{(\pi^* + r^*)p_M(p_M-\kappa\sigma(1+p_M)+1)^2}{2\kappa\sigma(p_M -1)} < 0
		\end{align*}
		Thus, for an increase in  $\phi_{\pi}$, $r^n_{LB}$ gets larger than $r^n_{B}$. 
	\end{proof}
	\begin{proposition}\label{prop:cPHIpilbcPHIpiub}
		Let $\phi_{\pi}\in(\phi_{\pi,LB},\phi_{\pi,UB})$. Then $r^n_{B} = \text{max}\left\{r^n_{LB}, r^n_{UB}, r^n_{B}\right\}$. 
	\end{proposition}
	\begin{proof}
		This follows directly from Propositions \ref{prop:cPHIpiub} and \ref{prop:cPHIpilb}. 
		To see that $r^n_{LB} < r^n_{B}$ recall that when $\phi_{\pi} = \phi_{\pi,UB}$,  $r^n_{LB} = r^n_{B}$ and the following relation holds:
		\begin{align*}
			\frac{\partial r^n_{LB}}{\partial \phi_{\pi}} &> 0\\
			\frac{\partial r^n_{B}}{\partial \phi_{\pi}} & < 0
		\end{align*}
		Thus for a decrease in $\phi_{\pi}$ we have $r^n_{LB} < r^n_{B}$. 
		To see that $r^n_{UB} < r^n_{B}$ recall that when $\phi_{\pi} = \phi_{\pi,LB}$,  $r^n_{UB} = r^n_{B}$ and the following relation holds:
		\begin{align*}
			\frac{\partial r^n_{UB}}{\partial \phi_{\pi}} &< 0\\
			\frac{\partial r^n_{B}}{\partial \phi_{\pi}} & > 0
		\end{align*}
		Thus for an increase in $\phi_{\pi}$ we have $r^n_{UB} < r^n_{B}$. 
	\end{proof}
	\begin{proposition}\label{prop:pi_tr_zero}
		Let $\phi_{\pi}\in(1,\phi_{\pi,LB})$. Then $\frac{\partial\pi_{TR}^{RSS}}{\partial r^n} = 0 $ 
	\end{proposition}
	\begin{proof}
		By proposition \ref{prop:cPHIpiub} we have that 
		\begin{align*}
			r^n_{UB} = \text{max}\left\{r^n_{LB}, r^n_{UB}, r^n_{B}\right\}
		\end{align*}
		Then, by Proposition \ref{prop:cSHOCKS}, for all shocks $r^n\le r^n_{UB}$, we have $\pi_{UB} \le \pi_{TR}^{RSS}$ and that $\pi_{TR}^{RSS} = \pi_{TR}^{DSS}$. Since no larger shock can exist, it follows that 
		\begin{align*}
			\frac{\partial\pi_{TR}^{RSS}}{\partial r^n} = 0
		\end{align*}
	\end{proof}
	\begin{proposition}\label{prop:pi_dr_zero}
		Let $\phi_{\pi}\in(\phi_{\pi,UB},\infty)$. Then $\frac{\partial\pi_{DR}^{RSS}}{\partial r^n} = 0 $ 
	\end{proposition}
	\begin{proof}
		By proposition \ref{prop:cPHIpilb} we have that 
		\begin{align*}
			r^n_{LB} = \text{max}\left\{r^n_{LB}, r^n_{UB}, r^n_{B}\right\}
		\end{align*}
		Then, by Proposition \ref{prop:cSHOCKS}, for all shocks $r^n\le r^n_{LB}$, we have $\pi_{DR}^{RSS}\le\pi_{LB} $ and that $\pi_{DR}^{RSS} = \pi_{DR}^{DSS}$. Since no larger shock can exist, it follows that 
		\begin{align*}
			\frac{\partial\pi_{DR}^{RSS}}{\partial r^n} = 0
		\end{align*}
	\end{proof}
	\begin{proposition}\label{prop:pi_dr_pos}
		Let $\phi_{\pi}\in(1,\phi_{\pi,UB})$ and $p_M < \frac{1-\kappa\sigma}{1-\kappa\sigma}$. Then $\frac{\partial\pi_{DR}^{RSS}}{\partial r^n} > 0 $ 
	\end{proposition} 
	\begin{proof}
		Begin by noticing that 
		\begin{align*}
			\frac{\partial \pi_{DR}^{RSS}}{\partial r^n} = -\frac{\kappa\sigma\phi_{\pi}(\kappa\sigma+1)(p_M-1)}{\phi_{\pi}(p_M-1) + \kappa\sigma\phi_{\pi}(1+p_M) + 2}
		\end{align*}
		The numerator of our fraction is always positive. Thus it suffices to show that the denominator is always greater than zero:
		\begin{align*}
			\phi_{\pi}(p_M-1) + \kappa\sigma\phi_{\pi}(1+p_M) + 2 &> 0\\
			\phi_{\pi}((p_M-1) + \kappa\sigma(1+p_M)) &> -2\\
			\phi_{\pi} &< \frac{-2}{(p_M-1) + \kappa\sigma(1+p_M)} = \phi_{\pi,UB} 	
		\end{align*}
		Where $(p_M-1) + \kappa\sigma(1+p_M)<0$ if and only if $p_M < \frac{1-\kappa\sigma}{1-\kappa\sigma}$, which it does, by assumption. 
	\end{proof}
	\begin{corollary}
		If $p_M > \frac{1-\kappa\sigma}{1-\kappa\sigma}$ then $\phi_{\pi,UB} = \infty$
	\end{corollary}
	\begin{proof}
		Let $p_M > \frac{1-\kappa\sigma}{1-\kappa\sigma}$ Evaluating $\phi_{\pi,UB}$ implies that $\phi_{\pi,UB} < 0$, which is impossible, given that $\phi_{\pi}>1$.
	\end{proof}
	\begin{proposition}\label{prop:pi_tr_neg}
		Let $\phi_{\pi}\in(\phi_{\pi,LB},\infty)$. Then $\frac{\partial\pi_{TR}^{RSS}}{\partial r^n} < 0 $ 
	\end{proposition} 
	\begin{proof}
		Begin by noticing that 
		\begin{align*}
			\frac{\partial \pi_{TR}^{RSS}}{\partial r^n} = \frac{2\kappa\sigma(p_M-1)}{\left[\phi_{\pi}(p_M - \kappa\sigma(1-p_M) + 1) - 2\right]\left[p_M + 1\right]} + \frac{\kappa\sigma(p_M-1)}{(p_M +1)(\kappa\sigma\phi_{\pi} + 1)}
		\end{align*}
		Notice that the second summand in our fraction is always negative. Thus, it suffices to show that the first summand is also negative. It is clear that the numerator is less than zero. To see when the denominator is greater than zero, we evaluate the following inequality: 
		\begin{align*}
			\left[\phi_{\pi}(p_M - \kappa\sigma(1-p_M) + 1) - 2\right]\left[p_M + 1\right] &> 0 \\
			\phi_{\pi}(p_M - \kappa\sigma(1-p_M) + 1) &> 2 \\
			\phi_{\pi} &> \frac{2}{p_M - \kappa\sigma(1-p_M) + 1}  = \phi_{\pi,LB}
		\end{align*} 	
	\end{proof}
	\begin{proposition}\label{prop:oneRSS}
		The unique largest possible shock to an economy ($\text{max}\left\{r^n_{LB}, r^n_{UB}, r^n_{B}\right\}$) implies that $\pi_{DR}^{RSS} = \pi_{TR}^{RSS}$.
	\end{proposition}
	\begin{proof}
		Assume, by way of contradiction, that $\text{max}\left\{r^n_{LB}, r^n_{UB}, r^n_{B}\right\}$ implies that $\pi_{DR}^{RSS} \ne \pi_{TR}^{RSS}$. Thus, the risk-adjusted Fisher relation intersects the Taylor Rule at two distinct points. Recall that there does not exist $r^n > \text{max}\left\{r^n_{LB}, r^n_{UB}, r^n_{B}\right\}$ because it implies that the risk-adjusted Fisher relation and Taylor rule do not intersect. This implies a jump discontinuity in our derivative of the RSS inflation with respect to shock size. However, on the specified intervals above, the derivatives of the RSS inflation values are continuous, and thus yields a contradiction.
	\end{proof}
	
	
\subsection{Results for moderate $\phi_{\pi}$ case}	

	\begin{proposition}
		Let $\phi_{\pi}\in(\phi_{\pi,LB},\phi_{\pi,UB})$. Then, for the maximum level of uncertainty, $\pi_{DR}^{RSS} = \pi_{TR}^{RSS} = \pi_{B}$. 
	\end{proposition}
	\begin{proof}
		By Proposition \ref{prop:cPHIpilbcPHIpiub}, we have that $r^n_B$ is the maximum level of uncertainty. The proof follows directly from Proposition \ref{prop:oneRSS}.
	\end{proof}
	
	\begin{figure}[!ht]
		\begin{center}
			\caption{Risky Steady States with $\phi_{\pi}\in(\phi_{\pi,LB},\phi_{\pi,UB})$}
			\includegraphics[width = 12cm ]{Figs/Fig2/RAFR_varycSHOCK.eps}\label{fig:RAFR_Baseline}
		\end{center}
	\end{figure}

	\begin{corollary}
		Let $\phi_{\pi}\in(\phi_{\pi,LB},\phi_{\pi,UB})$. Then an increase in uncertainty increases inflation and the policy rate in the Deflationary Regime and decreases them in the Target Regime.
	\end{corollary}
	\begin{proof}
		This follows directly from Proposition \ref{prop:pi_dr_pos} and \ref{prop:pi_tr_neg} which state that
		\begin{align*}
			\frac{\partial\pi_{TR}^{RSS}}{\partial r^n} < 0\\
			\frac{\partial\pi_{DR}^{RSS}}{\partial r^n} > 0
		\end{align*} 
	\end{proof}
	\begin{corollary}
		Let $\pi^* > -r^*/(1-\phi_{\pi})$  and let $\phi_{\pi}\in(\phi_{\pi,LB},\phi_{\pi,UB})$. Then, there exists $r^n_*$ such that the Deflationary Regime Risky Steady State inflation and policy rates are positive and zero, respectively.
	\end{corollary}
	\begin{proof}
		If $\pi^* = -r^*/(1-\phi_{\pi})$, then $\pi_B = 0$. Thus, at $r^n_B$ the Deflationary Regime Risky Steady State inflation and policy rates are both and zero. 
		Further, notice that 
		\begin{align*}
			\frac{\partial\pi_B}{\partial\pi^*} = -\frac{1-\phi_{\pi}}{\phi_{\pi}} > 0
		\end{align*}
		Then, for $\pi^* > -r^*/(1-\phi_{\pi})$, $\pi_B > 0$ and there exists $r^n_* \le r^n_B$ such that the Deflationary Regime Risky Steady State inflation and policy rates are positive and zero, respectively.
	\end{proof}
	
	\begin{figure}[!ht]
		\begin{center}
			\caption{Risky Steady States with Positive Inflation Target and  $\phi_{\pi}\in(\phi_{\pi,LB},\phi_{\pi,UB})$}
			\includegraphics[width = 12cm] {Figs/Fig5/RAFR_varycSHOCK.eps}\label{fig:RAFR_Baseline_inftarg}
		\end{center}
	\end{figure}

\subsection{Results for low $\phi_{\pi}$ case}	

	\begin{proposition}
		Let $\phi_{\pi}\in(1,\phi_{\pi,LB})$. Then, for the maximum level of uncertainty, $\pi_{DR}^{RSS} = \pi_{TR}^{RSS} = \pi_{TR}^{DSS}$. 
	\end{proposition}
	\begin{proof}
		By Proposition \ref{prop:cPHIpilb}, we have that $r^n_{UB}$ is the maximum level of uncertainty. The proof follows directly from Proposition \ref{prop:oneRSS}.
	\end{proof}
	
	\begin{figure}[!ht]
		\begin{center}
			\caption{Risky Steady States with  $\phi_{\pi}\in(1,\phi_{\pi,LB})$}
			\includegraphics[width = 12cm ]{Figs/Fig4/RAFR_cPHIpi_1dot1.eps}\label{fig:RAFR_smallcPHIpi}
		\end{center}
	\end{figure}
	
	\begin{corollary}
		Let $\phi_{\pi}\in(1,\phi_{\pi,LB})$. If the uncertainty is sufficiently high, the RSS policy rate in the Deflationary Regime is positive. 
	\end{corollary}
	\begin{proof}
		This follows directly from Proposition \ref{prop:pi_tr_zero} and \ref{prop:pi_dr_pos} which state that
		\begin{align*}
			\frac{\partial\pi_{TR}^{RSS}}{\partial r^n} = 0\\
			\frac{\partial\pi_{DR}^{RSS}}{\partial r^n} > 0
		\end{align*} 
		A shock $r^n_{*} \in (r^n_{B}, r^n_{UB})$ will yield a shock that RSS inflation in the Deflationary Regime is greater than $\pi_{B}$, which translates into a positive RSS policy rate in the Deflationary Regime.
	\end{proof}
	\begin{corollary}
		Let $\pi^* > 0$ and let $\phi_{\pi}\in(1,\phi_{\pi,LB})$. Then, there exists a shock $r^n_*$ such that the Deflationary Regime Risky Steady State inflation and policy rates are both positive. 
	\end{corollary}
	\begin{proof}
		If $\pi^* > 0$ then at $r^n = r^n_{UB}$, the Deflationary Regime Risky Steady State inflation and policy rates are equal to the Deterministic Steady States and are both positive. 
	\end{proof}
	
	\begin{figure}[!ht]
		\begin{center}
			\caption{Risky Steady States with Positive Inflation Target and  $\phi_{\pi}\in(1,\phi_{\pi,LB})$}
			\includegraphics[width = 12cm] {Figs/Fig6/RAFR_cPHIpi_1dot1.eps}\label{fig:RAFR_smallcPHIpi_inftarg}
		\end{center}
	\end{figure}

\subsection{Results for high $\phi_{\pi}$ case}	
	
	\begin{proposition}
		Let $\phi_{\pi}\in(\phi_{\pi,UB},\infty)$. Then, for the maximum level of uncertainty, $\pi_{DR}^{RSS} = \pi_{TR}^{RSS} = \pi_{DR}^{DSS}$
	\end{proposition}
	\begin{proof}
		By Proposition \ref{prop:cPHIpiub}, we have that $r^n_{LB}$ is the maximum level of uncertainty. The proof follows directly from Proposition \ref{prop:oneRSS}.
	\end{proof}
	
	\begin{figure}[!ht]
		\begin{center}
			\caption{Risky Steady States with  $\phi_{\pi}\in(\phi_{\pi,UB},\infty)$}
			\includegraphics[width = 12cm ]{Figs/Fig3/RAFR_cPHIpi_5.eps}\label{fig:RAFR_largecPHIpi}
		\end{center}
	\end{figure}
	
	\begin{corollary}
		Let $\phi_{\pi}\in(\phi_{\pi,UB},\infty)$. If the uncertainty is sufficiently high, the steady-state policy rate in the target equilibrium is at the ELB. 
	\end{corollary}
	\begin{proof}
		This follows directly from Proposition \ref{prop:pi_dr_zero} and \ref{prop:pi_tr_neg} which state that
		\begin{align*}
			\frac{\partial\pi_{TR}^{RSS}}{\partial r^n} < 0\\
			\frac{\partial\pi_{DR}^{RSS}}{\partial r^n} = 0
		\end{align*} 
		A shock $r^n_{*} \in (r^n_{B}, r^n_{LB})$ will yield a shock that RSS inflation in the Target Regime is less than $\pi_{B}$, which translates into a zero RSS policy rate in the Target Regime.
	\end{proof}
	\begin{corollary}
		Let $\pi^*> -r^*/(1-\phi_{\pi})$  and let $\phi_{\pi}\in(\phi_{\pi,UB},\infty)$. Then, there exists $r^n_*$ such that Target Regime Risky Steady State inflation and policy rate are positive and zero, respectively.
	\end{corollary}
	\begin{proof}
		If $\pi^* = -r^*/(1-\phi_{\pi})$, then $\pi_B = 0$. Thus, at $r^n_B$ the Target Regime Risky Steady State inflation and policy rates are both and zero. 
		Further, notice that 
		\begin{align*}
		\frac{\partial\pi_B}{\partial\pi^*} = -\frac{1-\phi_{\pi}}{\phi_{\pi}} > 0
		\end{align*}
		Then, for $\pi^* > -r^*/(1-\phi_{\pi})$, $\pi_B > 0$ and there exists $r^n_* \le r^n_{B}$ such that the Target Regime Risky Steady State inflation and policy rates are positive and zero, respectively.
	\end{proof}
	
	\begin{figure}[!ht]
		\begin{center}
			\caption{Risky Steady States with with Positive Inflation Target and  $\phi_{\pi}\in(\phi_{\pi,UB},\infty)$}
			\includegraphics[width = 12cm] {Figs/Fig7/RAFR_cPHIpi_5.eps}\label{fig:RAFR_largecPHIpi_inftarg}
		\end{center}
	\end{figure}		
	
	%==========================================================
	%==========================================================
	%==========================================================
	\section{Numerical results from an AR(1)-shock model}
	\label{S:NumericalResults}
	%\label{S:Discussion}
	
	\subsection{The Model}
	
	The results above are derived from a semi log-linear model with a three-state i.i.d. shock. In this section, the model is as follows:
	
	\begin{align}
	& y_{t} = \mathbb{E}_t\{y_{t+1}\} - \sigma\left[\hat{i}_t - \mathbb{E}_t\{\hat{\pi}_{t+1}\}-r_t^n\right] \label{eq:EE_ar1}\\
	& \hat{\pi}_{t} = \kappa y_t + \beta\mathbb{E}_t\{\hat{\pi}_{t+1}\}\label{eq:PC_ar1}\\
	& \hat{\pi}_t = \pi_t - \pi^*\\
	& \hat{i}_t = i_t - i_{DSS} \\
	& i_t = \text{max}\left[i_{ELB},r^* + \pi^* + \phi_{\pi}(\pi_t - \pi^*)\right]\label{eq:TR_ar1}\\
	& i_{DSS} = r^* + \pi^*
	\end{align}
	$y_t, \pi_t, i_t$ are the output gap, inflation rate, and nominal interest rate of our economy. Equation (\ref{eq:EE_ar1}) is the consumption Euler equation, equation (\ref{eq:PC_ar1}) is the standard New Keynesian Phillips Curve, and equation (\ref{eq:TR_ar1}) is the interest-rate feedback rule followed by the central bank. 
	
	 In this section, we relax our assumption about the demand shock and model it as an exogenous AR(1) shock process: 
	\begin{align}
	r^n_t = \rho r^n_{t-1} + \epsilon_{r^n}
	\end{align}
	We approximate the AR(1) process of the exogenous shock using Markov chains, via the Rouwenhorst approximation method. This allows us to determine the grid points and transition probabilities for the Markov chain. 
	
	A recursive equilibrium for this stylized, semi log-linear model is given by a set of policy functions $\{y(\cdot), \pi(\cdot), i(\cdot)\}$ that satisfies the equilibrium conditions above. The model is solved with linear algebra and the details of how the matrices are constructed are detailed in Appendix \ref{A:SolutionMethod}. Table \ref{tab:ParameterValues_AR1} lists the parameter values used for this exercise. 
	
	\begin{table}[!htp]
		{\small
			\begin{center}
				\caption{Parameter Values for the Stylized Model\label{tab:ParameterValues_AR1}}
				\vspace{-1.5em}
				\begin{tabular}{llc}
					\multicolumn{3}{c}{}\\
					Parameter & Description  & Parameter Value  \\
					\hline
					\hline
					$\beta$ & Discount rate & $\frac{1}{1+0.0025}$ \\
					$\sigma$ & Inverse intertemporal elasticity of substitution  & 1\\
					$\kappa$ & Slope of Phillips Curve & 0.02 \\
					$400\pi^*$ & Annualized Inflation target in the Taylor rule & 0\%\\
					$400r^*$ & Annualized Natural Rate of Interest & 1\%\\
					$\phi_{\pi}$ & Coefficient on inflation in the Taylor rule & $[2, 4, 10]$\\
					$i_{ELB}$    & Effective lower bound & 0\\
					\hline
					$\rho$ & AR(1) coefficient for the demand shock & $0.80$ \\
					$\sigma_{r^n}$ & standard deviation of shocks to demand shock & $[0, \sigma_{r^n_{max}}]$\\
					\hline
					\hline
				\end{tabular}
			\end{center}
		}
		\vspace{-0.5em}
	\end{table}

	\subsection{Results for moderate $\phi_{\pi}$ case}	
	Figure (\ref{fig:PFsModeratecPHIpi}) present the policy functions for output, inflation, and the policy rate for a model with a moderate value of $\phi_{\pi}$. The solid black line represents $r^n_{max}$, which is the maximum degree of uncertainty in the economy before a solution does not exist. The top row of figures are policy functions for the Target Regime and the bottom row are policy functions for the Deflationary Regime. As in our three-state i.i.d. shock, $r^n_{max}$ is associated at the point where the risky steady states converge to $\pi_B$. 
	
	\begin{figure}[!ht]
	\begin{center}
		\caption{Policy Functions: Moderate  $\phi_{\pi}$}
		\includegraphics[width = 14cm ]{Figs/Fig8/rouwenhorst_pfs_cSIGMAd_cPHIpi4.eps}\label{fig:PFsModeratecPHIpi}
	\end{center}
	\end{figure}
	
	Figure (\ref{fig:MomentsModeratecPHIpi}) present how the RSS output, inflation, and policy rates, the expected value for output, inflation, and policy rates, and the probability of being at the ELB change as the degree of uncertainty increases in the economy. The solid black line represents values for the Target Regime and the dashed black lines represent values for the Deflationary Regime. 
	
	\begin{figure}[!ht]
	\begin{center}
		\caption{Model Moments: Moderate  $\phi_{\pi}$}
		\includegraphics[width = 15cm ]{Figs/Fig9/rouwenhorst_moments_cPHIpi4.eps}\label{fig:MomentsModeratecPHIpi}
	\end{center}
	\end{figure}
		
	\subsection{Results for low $\phi_{\pi}$ case}	
	
	Figure (\ref{fig:PFsLowcPHIpi}) present the policy functions for output, inflation, and the policy rate for a model with a low value of $\phi_{\pi}$. The solid black line represents $r^n_{max}$, which is the maximum degree of uncertainty in the economy before a solution does not exist. The top row of figures are policy functions for the Target Regime and the bottom row are policy functions for the Deflationary Regime. As in our three-state i.i.d. shock, $r^n_{max}$ is associated at the point where the risky steady states converge to the Deterministic Steady State for the Target Regime. 
	
	\begin{figure}[!ht]
		\begin{center}
			\caption{Policy Functions: Low  $\phi_{\pi}$}
			\includegraphics[width = 14cm ]{Figs/Fig10/rouwenhorst_pfs_cSIGMAd.eps}\label{fig:PFsLowcPHIpi}
		\end{center}
	\end{figure}
	
	Figure (\ref{fig:MomentsLowcPHIpi}) present how the RSS output, inflation, and policy rates, the expected value for output, inflation, and policy rates, and the probability of being at the ELB change as the degree of uncertainty increases in the economy. The solid black line represents values for the Target Regime and the dashed black lines represent values for the Deflationary Regime.
	
	\begin{figure}[!ht]
		\begin{center}
			\caption{Model Moments: Low  $\phi_{\pi}$}
			\includegraphics[width = 15cm ]{Figs/Fig11/rouwenhorst_moments_cPHIpi2.eps}\label{fig:MomentsLowcPHIpi}
		\end{center}
	\end{figure}
	
	
	\subsection{Results for high $\phi_{\pi}$ case}	
	
	Figure (\ref{fig:PFsHighcPHIpi}) present the policy functions for output, inflation, and the policy rate for a model with a moderate value of $\phi_{\pi}$. The solid black line represents $r^n_{max}$, which is the maximum degree of uncertainty in the economy before a solution does not exist. The top row of figures are policy functions for the Target Regime and the bottom row are policy functions for the Deflationary Regime. As in our three-state i.i.d. shock, $r^n_{max}$ is associated at the point where the risky steady states converge to the Deterministic Steady State for the Deflationary Regime. 
	
	\begin{figure}[!ht]
		\begin{center}
			\caption{Policy Functions: High  $\phi_{\pi}$}
			\includegraphics[width = 14cm ]{Figs/Fig12/rouwenhorst_pfs_cSIGMAd_cPHIpi10.eps}\label{fig:PFsHighcPHIpi}
		\end{center}
	\end{figure}
	
	Figure (\ref{fig:MomentsHighcPHIpi}) present how the RSS output, inflation, and policy rates, the expected value for output, inflation, and policy rates, and the probability of being at the ELB change as the degree of uncertainty increases in the economy. The solid black line represents values for the Target Regime and the dashed black lines represent values for the Deflationary Regime.
	
	\begin{figure}[!ht]
		\begin{center}
			\caption{Model Moments: High  $\phi_{\pi}$}
			\includegraphics[width = 15cm ]{Figs/Fig13/rouwenhorst_moments_cPHIpi10.eps}\label{fig:MomentsHighcPHIpi}
		\end{center}
	\end{figure}
			
	%==========================================================
	%==========================================================
	%==========================================================
	%==========================================================
	%==========================================================
	%==========================================================
	%==========================================================
	%==========================================================
	%========================================================== Discussion
	%==========================================================
	%==========================================================
	%==========================================================
	%==========================================================
	%==========================================================
	%==========================================================
	%==========================================================	
	\section{Discussion}
	\label{S:Discussion}
	
	\subsection{Relation with Mertens and Williams (2018)}
	
	
	
	\subsection{Relation with Armenter (2017)}
	
	
	
	%==========================================================
	%==========================================================
	%==========================================================
	%==========================================================
	%==========================================================
	%==========================================================
	%==========================================================
	%==========================================================
	%========================================================== Conclusion
	%==========================================================
	%==========================================================
	%==========================================================
	%==========================================================
	%==========================================================
	%==========================================================
	%==========================================================
	\section{Conclusion}
	\label{S:Conclusion}
	
	
	
	\newpage
	\bibliographystyle{econometrica}
	\bibliography{All}
	
	
	%==========================================================
	%==========================================================
	%==========================================================
	%==========================================================
	%==========================================================
	%==========================================================
	%==========================================================
	%==========================================================
	%==========================================================
	%==========================================================
	%==========================================================
	%==========================================================
	%==========================================================
	%==========================================================
	%==========================================================
	%========================================================== Apepndix
	%==========================================================
	%==========================================================
	%==========================================================
	%==========================================================
	%==========================================================
	%==========================================================
	%==========================================================
	%==========================================================
	%==========================================================
	%==========================================================
	%==========================================================
	%==========================================================
	%==========================================================
	%==========================================================
	\newpage
	\appendix
	\setcounter{equation}{0}
	\renewcommand{\theequation}{\thesection\arabic{equation}}
	\begin{center}
		\textbf{\LARGE{Technical Appendix for Online Publication}}
	\end{center}
	
	\begin{singlespace}
		
		\vspace{2em}
		\noindent This technical appendix is organized as follows:
%		\begin{itemize}
			%\item Section~\ref{A:pTpD} analyzes how $p_{T}$ and $p_{D}$ affect the allocations in the recursive sunspot equilibrium.
			%\item Section~\ref{A:SolutionAlgorithm} explains the solution algorithm for the empirical model.
%		\end{itemize}
		
		\section{Three-State i.i.d. Shock Model}\label{A:ThreeStateiid}
		\subsection{Algebraic Derivation of Risk Adjusted Fisher Relation}
		
		We present the full algebraic derivation of the piecewise risk-adjusted Fisher relation. In our three-period model, there are two cases to consider: 
		\begin{enumerate}
			\item[(i)] the risk-adjusted Fisher relation when the ELB binds in states $L$ and $M$.
			\item[(ii)] the risk-adjusted Fisher relation when the ELB binds in state $L$ only.
		\end{enumerate}
		Recall the following system of equations: 
		the following system of equations: 
		\begin{align}
		y_{H} & = \mathbb{E}\{y_{t+1}\} - \sigma\left[i_{H} - i_{DSS} - \mathbb{E}\{\pi_{t+1} - \pi^*\}-r^n\right] \label{eq:EE_H_appx} \\
		y_{M} & = \mathbb{E}\{y_{t+1}\} - \sigma\left[i_{M} - i_{DSS} - \mathbb{E}\{\pi_{t+1} - \pi^*\}\right] \label{eq:EE_M_appx}\\
		y_{L} & = \mathbb{E}\{y_{t+1}\} - \sigma\left[i_{L} - i_{DSS} - \mathbb{E}\{\pi_{t+1} - \pi^*\}+r^n\right] \label{eq:EE_L_appx}\\
		\pi_{H} - \pi^* & = \kappa y_{H} + \beta\mathbb{E}\{\pi_{t+1} - \pi^*\} \label{eq:PC_H_appx}\\
		\pi_{M} - \pi^* & = \kappa y_{M} + \beta\mathbb{E}\{\pi_{t+1} - \pi^*\} \label{eq:PC_M_appx}\\
		\pi_{L} - \pi^* & = \kappa y_{L} + \beta\mathbb{E}\{\pi_{t+1} - \pi^*\} \label{eq:PC_L_appx}\\
		i_{H} & = \text{max}\left[0, r^* + \pi^* +  \phi_{\pi}(\pi_{H} - \pi^*)\right] \label{eq:TR_H_appx}\\
		i_{M} & = \text{max}\left[0, r^* + \pi^* + \phi_{\pi}(\pi_{M} - \pi^*)\right] \label{eq:TR_M_appx}\\
		i_{L} & = \text{max}\left[0, r^* + \pi^* + \phi_{\pi}(\pi_{L} - \pi^*)\right] \label{eq:TR_L_appx}
		\end{align}
		where $\mathbb{E}\{x_{t+1}\} \coloneqq \frac{1-p_M}{2}x_H + p_Mx_M + \frac{1-p_M}{2}x_L$ and $x\in\{y,\pi\}$.
		
		\subsubsection{ELB Binds in States $L$ and $M$}
		
		Recall the system of equations given by equations (\ref{eq:EE_H_appx} - \ref{eq:TR_L_appx}). We will show that all policy functions, and thus all expectations, can be written as a function of $\pi_M$. We assume that $i_L = 0$ and $i_M = 0$.
		
		Begin by noticing that equations (\ref{eq:EE_H_appx}) and (\ref{eq:EE_L}) can be rewritten as functions of $y_M$:
		\begin{align}
		& y^{LM}_{H} = g_1(y^{LM}_M) = y^{LM}_M - \sigma\left(r^* + \pi^* \phi_{\pi}(\pi^{LM}_H - \pi^*) - r^n\right) \\
		& y^{LM}_{L} = g_2(y^{LM}_M)= y^{LM}_M - \sigma r^n 
		\end{align}
		Next, given our simplified Euler Equations, we will simplify the Phillips curves. Equations (\ref{eq:PC_H_appx}) and (\ref{eq:PC_L_appx}) can all be rewritten as functions of $\pi_M$:
		\begin{align}
		& \pi^{LM}_{H} = f_1(\pi_M) =  \frac{\pi_M -\kappa\sigma(r^* + \pi^*(1-\phi_{\pi})- r^n)}{1+\kappa\sigma\phi_{\pi}}\\
		& \pi^{LM}_{L} = f_2(\pi^{LM}_M) = \pi_M -\kappa\sigma r^n
		\end{align}
		From here, it is easy to see that $\mathbb{E}_{LM}\{\pi_{t+1}\}$ is a function of $\pi_M$. Recall that $\mathbb{E}_{LM}\{\pi_{t+1}\} \coloneqq \frac{1-p_M}{2}\pi^{LM}_H + p_M\pi_M + \frac{1-p_M}{2}\pi^{LM}_L = h_1(\pi_M)$, which is the sum of functions of $\pi_M$.Finally, we show that $y^{LM}_M$ and $\mathbb{E}_{LM}\{y_{t+1}\}$ is a function of $\pi_M$. The first statement follows from the functional relationship of equation (\ref{eq:PC_M_appx}): thus $y^{LM}_M = f_4(\pi_M)$. The second statement follows similarly to why $\mathbb{E}_{LM}\{\pi_{t+1}\}$ is a function of $\pi_M$: $\mathbb{E}_{LM}\{y_{t+1}\} \coloneqq \frac{1-p_M}{2}y^{LM}_H + p_My^{LM}_M + \frac{1-p_M}{2}y^{LM}_L = g_3(y_M) = g_3(f_4(\pi_M)) = h2(\pi_M)$. This is the sum of functions of $y^{LM}_M$, which is a function of $\pi_M$.
		
		Thus, for $\pi_{LB} \le \pi_M \le \pi_B$, the risk adjusted Fisher relation can be written in the following way:
		\begin{align}
			i_M = r^* + \pi_M + \sigma^{-1}\left(\mathbb{E}_{LM}\{y_{t+1}\} - y_M^{LM}\right) + \left(\mathbb{E}_{LM}\{\pi_{t+1}\} - \pi_M^{LM}\right)
		\end{align}
		
		\subsubsection{ELB Binds in State $L$}
		
		We will show that all policy functions, and thus all expectations, can be written as a function of $\pi_M$. We assume that $i_L = 0$ only.
		
		Begin by noticing that equations (\ref{eq:EE_H_appx}) and (\ref{eq:EE_L_appx}) can be rewritten as functions of $y_M$:
		\begin{align}
		& y^{L}_{H} = g_1(y^{L}_M) = y^{L}_M - \sigma\left(\phi_{\pi}(\pi_M - \pi_{H}) + r^n\right) \\
		& y^{L}_{L} = g_2(y^{L}_M)= y^{L}_M + \sigma\left(\phi_{\pi}(\pi_M - \pi^*) + r^* + \pi^* - r^n \right) 
		\end{align}
		Next, given our simplified Euler Equations, we will simplify the Phillips curves. Equations (\ref{eq:PC_H_appx}) and (\ref{eq:PC_L_appx}) can all be rewritten as functions of $\pi_M$:
		\begin{align}
		& \pi^{L}_{H} = f_1(\pi_M) =  \frac{\pi_M(1+ \kappa\sigma\phi_{\pi}) + \kappa\sigma r^*}{1+\kappa\sigma\phi_{\pi}}\\
		& \pi^{L}_{L} = f_2(\pi^{L}_M) = \pi_M(1+\kappa\sigma\phi_{\pi}) + \kappa\sigma(-\phi_{\pi}\pi^* + r^* + \pi^* - r^n)
		\end{align}
		From here, it is easy to see that $\mathbb{E}_{L}\{\pi_{t+1}\}$ is a function of $\pi_M$. Recall that $\mathbb{E}_{L}\{\pi_{t+1}\} \coloneqq \frac{1-p_M}{2}\pi^{L}_H + p_M\pi_M + \frac{1-p_M}{2}\pi^{L}_L = h_1(\pi_M)$, which is the sum of functions of $\pi_M$.Finally, we show that $y^{L}_M$ and $\mathbb{E}_{L}\{y_{t+1}\}$ is a function of $\pi_M$. The first statement follows from the functional relationship of equation (\ref{eq:PC_M_appx}): thus $y^{L}_M = f_4(\pi_M)$. The second statement follows similarly to why $\mathbb{E}_{L}\{\pi_{t+1}\}$ is a function of $\pi_M$: $\mathbb{E}_{L}\{y_{t+1}\} \coloneqq \frac{1-p_M}{2}y^{L}_H + p_My^{L}_M + \frac{1-p_M}{2}y^{L}_L = g_3(y_M) = g_3(f_4(\pi_M)) = h2(\pi_M)$. This is the sum of functions of $y^{L}_M$, which is a function of $\pi_M$.
		
		Thus, for $\pi_{B} \le \pi_M \le \pi_{UB}$, the risk adjusted Fisher relation can be written in the following way:
		\begin{align}
		i_M = r^* + \pi_M + \sigma^{-1}\left(\mathbb{E}_{L}\{y_{t+1}\} - y_M^{L}\right) + \left(\mathbb{E}_{L}\{\pi_{t+1}\} - \pi_M^{L}\right)
		\end{align}
		
		\subsection{Extension of Analytical Results}
		
		The results found in the main body of the paper only consider a unique maximum uncertainty value of the set of shock sizes $\text{max}\left\{r^n_{LB}, r^n_{UB}, r^n_{B}\right\}$. However, it is possible that there is no unique maximum of the set. Under this scenario, $\pi_{DR}^{RSS} \ne \pi_{TR}^{RSS}$. However, our economy has an infinite number of risky steady states. 
		
		To intuitively understand what is happening, recall that when $r^n = r^n_{B}$, this is the size of the shock such that $\pi_{TR}^{RSS} = \pi_B$ or  $\pi_{DR}^{RSS} = \pi_B$. Similarly, if $r^n = r^n_{UB}$, then $\pi_{TR}^{RSS} = \pi_{TR}^{DSS}$. Thus, if $r^n_{B} = r^n_{UB} = \text{max}\left\{r^n_{LB}, r^n_{UB}, r^n_{B}\right\}$, a scenario occurs where the risk-adjusted Fisher relation falls exactly along the Taylor rule. 
		
		This infinite number of Risky Steady States, coincidentally, occurs at a specific value of $\phi_{\pi}$: $\phi_{\pi, LB}$ for the case considered in the above paragraph and $\phi_{\pi,UB}$ for the case when  $r^n_{B} = r^n_{LB} = \text{max}\left\{r^n_{LB}, r^n_{UB}, r^n_{B}\right\}$. Each case is graphically detailed in Figure (\ref{fig:cPHIpilb}) and (\ref{fig:cPHIpiub}) respectively. 
		
		\begin{figure}[!ht]
			\begin{center}
				\caption{Risk Adjusted Fisher Relation $\phi_{\pi} = \phi_{\pi,LB}$}
				\includegraphics[width = 12cm ]{Figs/Appendix/Fig1/RAFR_cPHIpi_lb.eps}\label{fig:cPHIpilb}
			\end{center}
		\end{figure}
		
		\begin{figure}[!ht]
			\begin{center}
				\caption{Risk Adjusted Fisher Relation $\phi_{\pi} = \phi_{\pi,UB}$}
				\includegraphics[width = 12cm ]{Figs/Appendix/Fig2/RAFR_cPHIpi_ub.eps}\label{fig:cPHIpiub}
			\end{center}
		\end{figure}
		
		
		\section{Details on the Solution Method}
		\label{A:SolutionMethod}
		\setcounter{equation}{0}
		\subsection{Model with Rouwenhorst Approximation to AR(1) shock}
		
		Recall that the problem is to find a set of $\{y(\cdot), \hat{\pi}(\cdot), \hat{i}(\cdot), \pi(\cdot), i(\cdot)\}$ that satisfies the equilibrium conditions: 
		\begin{align}
		& y_{t} = \mathbb{E}_t\{y_{t+1}\} - \sigma\left[\hat{i}_t - \mathbb{E}_t\{\hat{\pi}_{t+1}\}-r_t^n\right] \label{eq:EE_appx}\\
		& \hat{\pi}_{t} = \kappa y_t + \beta\mathbb{E}_t\{\hat{\pi}_{t+1}\}\label{eq:PC_appx}\\
		& \hat{\pi}_t = \pi_t - \pi^* \label{eq:pihat_appx}\\
		& \hat{i}_t = i_t - i_{DSS}  \label{eq:ihat_appx}\\
		& i_t = \text{max}\left[i_{ELB},r^* + \pi^* + \phi_{\pi}(\pi_t - \pi^*)\right]\label{eq:TR_appx}\\
		& i_{DSS} = r^* + \pi^* \label{eq:idss_appx}
		\end{align}
		
		Substituting out $\hat{\pi}_t$, $\hat{i}_t$, and $i_{DSS}$ using equations (\ref{eq:pihat_appx}), (\ref{eq:ihat_appx}), and (\ref{eq:idss_appx}), this system can be reduced to a system of three functional equations for $y(\cdot),\pi(\cdot), i(\cdot)$. 
		
		\begin{align}
		y_{t} &= \mathbb{E}_t\{y_{t+1}\} - \sigma\left[i_t - (r^* + \pi^*) - \mathbb{E}_t\{\pi_{t+1} - \pi^*\}-r_t^n\right] \\
		\pi_t - \pi^* &= \kappa y_t + \beta\mathbb{E}_t\{\hat{\pi}_{t+1}\}\\
		i_t &= \text{max}\left[i_{ELB},r^* + \pi^* + \phi_{\pi}(\pi_t - \pi^*)\right]
		\end{align}
		
		Consider an $n$-state discretization of an AR(1) shock approximated via the Rouwenhorst method. The Rouwenhorst approximation method will yield an $n \times 1$ vector of grid points $[r_1^n,\dots,r_n^n]$ and an $n \times n$ matrix, $T$, of transition probabilities: 
		\begin{align}
			T=
			\begin{bmatrix}
			p_{1,1} & p_{1,2} & \dots & p_{1,n} \\
			p_{2,1} & p_{2,2} & \dots & p_{2,n} \\
			\vdots & \vdots & \ddots & \vdots \\
			p_{n,1} & p_{n,2} & \dots & p_{n,n} \\
			\end{bmatrix}
			= 
			\begin{bmatrix}
			t_1 \\
			t_2 \\
			\vdots \\
			t_n  \\
			\end{bmatrix}
		\end{align}
		where $t_i$ is the $i^{th}$ row of $T$.
		
		Given this  $n$-state discretization, we are left with a series of $n$ equations and $n$ unknowns to solve for: 

		\begin{align*}
		& y_{1} = \mathbb{E}_1\{y_{t+1}\} - \sigma\left[i_1 - (r^* + \pi^*) - \mathbb{E}_1\{\pi_{t+1} - \pi^*\}-r_1^n\right] \\
		&\vdots \nonumber\\
		& y_{n} = \mathbb{E}_n\{y_{t+1}\} - \sigma\left[i_n - (r^* + \pi^*) - \mathbb{E}_n\{\pi_{t+1} - \pi^*\}-r_n^n\right]\\ 
%		\end{align*}
%		\begin{align*}
		& \pi_1 - \pi^* = \kappa y_1 + \beta\mathbb{E}_1\{\hat{\pi}_{t+1}\}\\
		&\vdots \nonumber\\
		& \pi_n - \pi^* = \kappa y_n + \beta\mathbb{E}_n\{\hat{\pi}_{t+1}\}\\
%		\end{align*}
%		\begin{align*}
		& i_1 = \text{max}\left[i_{ELB},r^* + \pi^* + \phi_{\pi}(\pi_1 - \pi^*)\right]\\
		\vdots \nonumber\\
		& i_n = \text{max}\left[i_{ELB},r^* + \pi^* + \phi_{\pi}(\pi_n - \pi^*)\right]
		\end{align*}
		Here, $\mathbb{E}_i\{\cdot\}$ is the conditional expectation of our policy function, given state $i$. It is formally defined as the $t_i\cdot z$, where $z = \{z_1,\cdots,z_n\}^{T}$, for a given policy function. 
		
		Notice that, absent the ELB constraint, we are left with a linear-system of equations and can be solved for using basic matrix algebra. Let $A$ be a matrix of coefficients, $x$ be a vector of variables, $b$ be a vector of coefficients, where 
		\begin{align*}
		\begin{array}{ccc}
		A = 
		\begin{bmatrix}
			A_{1,1} & A_{1,2} & A_{1,3} \\
			A_{2,1} & A_{2,2} & A_{2,3} \\
			A_{3,1} & A_{3,2} & A_{3,3} \\
			\end{bmatrix} & 
			x =
			\begin{bmatrix}
			y_1\\
			\vdots\\
			y_n\\
			\pi_1\\
			\vdots\\
			\pi_n\\
			i_1\\
			\vdots\\
			i_n\\
			\end{bmatrix}& 
			b =
			\begin{bmatrix}
			r^* + \sigma\pi^* + r^n_1\\
			\vdots\\
			r^* + \sigma\pi^* + r^n_n\\
			-\beta\pi^*\\
			\vdots\\
			-\beta\pi^*\\
			r^* + \pi^*(1-\phi_{\pi})\\
			\vdots\\
			r^* + \pi^*(1-\phi_{\pi})\\			
			\end{bmatrix}
		\end{array}
		\end{align*}
		and
		
		\begin{align*}
		\begin{array}{lll}
			A_{1,1} = 
			\begin{bmatrix}
				1-p_{1,1} & p_{1,2} & \cdots & p_{1,n} \\
				p_{2,1} & 1-p_{2,2} & \cdots & p_{2,n} \\
				\vdots & \vdots & \ddots & \vdots \\
				p_{1,1} & p_{n,2} & \cdots & 1-p_{n,n} \\
			\end{bmatrix} & 
			A_{1,2} = -\sigma\cdot T& 
			A_{1,3} = 
			\begin{bmatrix}
				\sigma & 0 & \cdots & 0 \\
				0 & \sigma & \cdots & 0 \\
				\vdots & \vdots & \ddots & \vdots \\
				0 & 0 & \cdots & \sigma \\
			\end{bmatrix}\\
			A_{2,1} = 
			\begin{bmatrix}
				\kappa & 0 & \cdots & 0 \\
				0 & \kappa & \cdots & 0 \\
				\vdots & \vdots & \ddots & \vdots \\
				0 & 0 & \cdots & \kappa \\
			\end{bmatrix}& 
			A_{2,2} = \beta A_{1,1}& 
			A_{2,3} = 0\cdot A_{1,1}\\
			A_{3,1} = A_{2,3}& 
			A_{3,2} = 
			\begin{bmatrix}
				-\phi_{\pi} & 0 & \cdots & 0 \\
				0 & -\phi_{\pi} & \cdots & 0 \\
				\vdots & \vdots & \ddots & \vdots \\
				0 & 0 & \cdots & -\phi_{\pi} \\
			\end{bmatrix}& 
			A_{3,3} = 
			\begin{bmatrix}
				1 & 0 & \cdots & 0 \\
				0 & 1 & \cdots & 0 \\
				\vdots & \vdots & \ddots & \vdots \\
				0 & 0 & \cdots & 1 \\
			\end{bmatrix}
		\end{array}
		\end{align*}
		
		There are two algorithms to consider. First, we consider the algorithm that solves for the policy functions of the Target Regime. Then we consider the algorithm that solves for the policy functions of the Deflationary Regime.
		
		\subsection{Algorithm for the Target Regime}
		The algorithm to solve for the policy functions in the Target Regime is as follows. Start by assuming that the ELB does not bind in any period and solve the linear system of equations. If $i_n < 0$ then assume that $i_n = 0$ and resolve the system of equations. If $i_{n-1} < 0$ then assume that $i_{n-1} = 0$ and resolve the system of equations. Continue this process until for all $j\in(1,\dots,n)$, $i_j \ge 0$. 
		
		\subsection{Algorithm for the Deflationary Regime}
		The algorithm to solve for the policy functions in the Deflationary Regime is as follows. Start by assuming that the ELB binds in all period and solve the linear system of equations. If the implied interest rate, $i_1^{imp} \equiv r^* + \pi^* + \phi_{\pi}(\pi_1 - \pi^*) > i_{ELB}$ then assume that $i_i \ne i_{ELB}$ and resolve the system of equations. If $i_{2} > i_{ELB}$ then assume that $i_2^{imp} > i_{ELB}$ and resolve the system of equations. Continue this process until for all $j\in(1,\dots,n)$, $i_j^{imp} > 0$. 
		
		\section{Analytical results in models with a continuous-state shock}
		
		\subsection{Uniform distribution (i.i.d)}
		
		
		\subsection{Normal distribution (i.i.d)}
		
		
		\section{Risk-Adjusted Fisher Relation with non-iid shocks}
		
		\subsection{Three-state Markov shock}
		
		
		\subsection{AR(1) shock}
		
		
		
	\end{singlespace}
	
	
\end{document}




