\documentclass[11pt]{article}
\usepackage{scrextend}
\usepackage{amssymb}
\usepackage{amsfonts}
\usepackage{amsmath}
\usepackage{mathtools}
\usepackage{bm}

%\usepackage{dsfont}
% \usepackage{bbm}

\usepackage[nohead]{geometry}
\usepackage[onehalfspacing]{setspace}
\usepackage[bottom]{footmisc}
\usepackage{indentfirst}
\usepackage{endnotes}
\usepackage{mathtools}

\usepackage{graphicx}
\usepackage{subcaption}
\usepackage{graphics}
\usepackage{epstopdf}


%\usepackage{epsfig}

\usepackage{lscape}
\usepackage{titlesec}
\usepackage{array}

%\usepackage{hyperref}

\usepackage{flexisym}
\usepackage{nccfoots}
\usepackage{datetime}
\usepackage{multirow}
\usepackage{booktabs}
\usepackage{rotating}

\usepackage{amsthm}
\newtheorem{theorem}{Theorem}
\newtheorem{definition}{Definition}
\newtheorem{remark}{Remark}
\newtheorem{proposition}{Proposition}
\newtheorem{corollary}{Corollary}[proposition]
\newtheorem{lemma}{Lemma}[theorem]

\usepackage{chngcntr}
\usepackage{apptools}
\AtAppendix{\counterwithin{proposition}{section}}
\AtAppendix{\counterwithin{definition}{section}}
\AtAppendix{\counterwithin{corollary}{section}}


\usepackage{thmtools}
\usepackage{thm-restate}

\usepackage[usenames,dvipsnames]{color}

\usepackage[longnamesfirst]{natbib}
\usepackage[justification=centering]{caption}

%\usepackage{datetime}

\DeclarePairedDelimiter\abs{\lvert}{\rvert}%
\DeclarePairedDelimiter\norm{\lVert}{\rVert}%

\definecolor{darkgray}{gray}{0.30}

%\usepackage[dvips, colorlinks=true, linkcolor=darkgray,

\usepackage[colorlinks=true, linkcolor=darkgray, citecolor=darkgray, urlcolor=darkgray, bookmarks=false, ,
pdfstartview={FitV},
pdftitle={Zero Bound Risk},
pdfauthor={Taisuke Nakata},
pdfkeywords={Liquidity Trap, Zero Lower Bound}]{hyperref}
\usepackage{xcolor,colortbl}
\usepackage{float}

\newcommand*{\LargerCdot}{\raisebox{-.5ex}{\scalebox{2}{$\cdot$}}}

\makeatletter
\def\@biblabel#1{\hspace*{-\labelsep}}
\makeatother
\geometry{left=1.2in,right=1.2in,top=1in,bottom=1in}


\begin{document}

	\title{Risk Adjusted Fisher Relation in Nonlinear NK Model with AR$(1)$ Shock }
	\author{
		Philip Coyle}
	\newdateformat{mydate}{\monthname[\THEMONTH] \THEYEAR}
	%\newdateformat{mydate}{This Draft: \monthname[\THEMONTH] \THEYEAR}
	\date{\mydate\today}

	\maketitle

	\vspace{-0.3in}


\section{Problem with the model}

I think there is an issue with equation $52$. The equation is as follows:
\begin{align*}
	C_i = C_{mid}\left(\frac{\delta_i R_i}{R_{mid}}\right)^{-\frac{1}{\chi_c}}
\end{align*}
The construction of the ''Demeaned Euler Equation'' is slightly different than that from the iid case, due to the persistence of the shock.

Given the EE, we can rewrite it as follows:
\begin{align}
C_{t}	= \left(\beta \delta_t R_t \mathbb{E}_t \left\{C_{t+1}^{\chi_c} \Pi_{t+1}^{-1}\right\}\right)^{-\left(\frac{1}{\chi_c}\right)}
\end{align}
Then, the nonlinear equivalent demeaned EE is:
\begin{align}
	\frac{C_{RSS}(\delta \neq \delta_{mid})}{C_{RSS}(\delta = \delta_{mid})} & = \left[\frac{\beta \delta(\neq \delta_{mid}) R_{RSS}(\delta \neq \delta_{mid}) \mathbb{E} \left\{C_(\delta' | \delta \neq \delta_{mid})^{-\chi_c} \Pi(\delta' | \delta \neq \delta_{mid})^{-1}\right\}}{\beta \delta_{mid} R_{RSS}(\delta = \delta_{mid}) \mathbb{E} \left\{C(\delta' | \delta = \delta_{mid})^{-\chi_c} \Pi(\delta' | \delta = \delta_{mid})^{-1}\right\}}\right]^{-\left(\frac{1}{\chi_c}\right)} \\
	& = \left[\frac{\delta(\neq \delta_{mid}) R_{RSS}(\delta \neq \delta_{mid}) \mathbb{E} \left\{C_(\delta' | \delta \neq \delta_{mid})^{-\chi_c} \Pi(\delta' | \delta \neq \delta_{mid})^{-1}\right\}}{R_{RSS}(\delta = \delta_{mid}) \mathbb{E} \left\{C(\delta' | \delta = \delta_{mid})^{-\chi_c} \Pi(\delta' | \delta = \delta_{mid})^{-1}\right\}}\right]^{-\left(\frac{1}{\chi_c}\right)}
\end{align}



These expectation terms \emph{don't} cancel out, due to the persistence of the shock. In other words, tomorrow's state depends on today's state.

The incorrect demeaned EE is what I believe is causing the issue. Now I understand when programming in the iid shock, the results look similar -- because the expectations cancel out

\end{document}
